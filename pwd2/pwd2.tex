\documentclass{article}
\setlength{\headheight}{8pt}

\usepackage[a4paper, margin=0.3in]{geometry}
\usepackage{lmodern}
\usepackage[polish]{babel}
\usepackage[T1]{fontenc}
\usepackage{amsmath, amssymb, amsthm}
\usepackage{hyperref}
\usepackage[shortlabels]{enumitem}
\usepackage{physics}
\usepackage{multicol}
\usepackage{titlesec}
\usepackage{tikz}
\usetikzlibrary{automata, positioning}

% Definicje środowisk do twierdzeń, lematów, dowodów itp.
\newtheorem{theorem}{Twierdzenie}[section]
\newtheorem{lemma}[theorem]{Lemat}
\newtheorem{corollary}[theorem]{Wniosek}
\newtheorem{proposition}[theorem]{Stwierdzenie}
\theoremstyle{definition}
\newtheorem{definition}{Definicja}[section]
\theoremstyle{remark}
\newtheorem*{remark}{Uwaga}
\newtheorem*{example}{Przykład}
\newtheorem*{property}{Własność}

\title{Praca domowa 2}
\author{Marcin Szopa 459531}
\date{\today}

% Remove padding at the top of subsections
\titlespacing*{\subsection}{0pt}{*0}{*0}

% Redefine section numbering to use letters
\renewcommand{\thesection}{\Alph{section}}

% Define a command for singleton sets
\newcommand{\singleton}[1]{\left\{ #1 \right\}}

\begin{document}

\maketitle

\section{}\label{sec:a}

\begin{remark}
    Zauważmy na początku, że jako że \(\Sigma\) jst językiem jednoliterowym, to
    \(\abs{w} = \#_a(w)\).
\end{remark}

Są 2 typy takich wielomianów:

\begin{enumerate}
    \item \(ax + b, a,b \in Z\),
    \item \(a_nx^n + a_{n-1}x^{n-1} + \cdots + a_1x + a_0, a_i \in Z \land a_n < 0\).
\end{enumerate}

\subsection{liniowy}

Weźmy dowolny wielomian postaci \( p(x)=bx + c\), gdzie \(b,c \in Z\). 
Niech \(A_p := \mathcal{L}(a^x \$ a^{bx + c})\) dla \(x \in \mathcal{N}\).
\(A_p\) jest językiem bezkontekstowym, bo można go opisać za pomocą gramatyki
o symbolu startowym \(S\) i produkcjach:
\begin{align*}
    S &\to a S a^b  \\
    S &\to \$ a^c
\end{align*}
Prosto pokazać, że \(L_p\) jest bezkontekstowy, bo \(A_p = L_p\).

\begin{proof}
    `\(\subseteq\)'
    Niech \(w \in A_p\). Stąd \(w = a^x \$ a^{bx + c}\) dla pewnego \(x \in \mathcal{N}\).
    \( \abs{a^x}=x, \abs{a^{bx + c}}=bx + c = p(x)\). Zatem \(w \in L_p\).
\end{proof}

\begin{proof}
    `\(\supseteq\)'
    Niech \(w \in L_p\). Stąd \(w = u \$ v\), że \( \abs{v} = p(\abs{u})
    = bx+c\). Zatem \(w = a^x \$ a^{bx + c} \in A_p\).
\end{proof}

Zatem \(L_p\) jest bezkontekstowy.

\subsection{o współczynniku przy najwyższej potędzie ujemnym}

Weźmy dowolny wielomian postaci \( p(x)=a_n x^n + a_{n-1}x^{n-1} + \cdots + a_1x + a_0\), gdzie \(a_i \in Z \land a_n < 0\).

Zauważmy, że \(\left\{ x: 0 \leq p(x) \right\}\) jest skończony, bo d.d.d. \(n\) \(a_n x^n < a_{n-1}x^{n-1} + \cdots + a_1x + a_0\) dla \(x \geq 0\).

Zatem \(L_p\) jest językiem skończonym, bo \(p(x)\) ma skończoną liczbę argumentów
dla których przyjmuje wartości nieujemne.
Skoro \(L_p\) jest językiem skończonym, to również regularny, a zatem bezkontekstowy.

\subsection{pozostałe przypadki}\label{sec:a:remaining}

Udowodnimy, że w pozostałych przypadkach \(L_p\) nie jest bezkontekstowy.

\begin{proof}

Niech \(p(x)=a_n x^n + a_{n-1}x^{n-1} + \cdots + a_1x + a_0\), gdzie \(a_i \in Z\) i \(a_n > 0 \land n>1\).
Zauważmy, że \(\left\{ x: 0 \leq p(x) \right\}\) jest nieskończony,
bo d.d.d. \(x\) \(a_n x^n > a_{n-1}x^{n-1} + \cdots + a_1x + a_0\).

Załóżmy, że \(L_p\) jest bezkontekstowy. Weźmy dowolne \(c \in L_p, n \in \mathcal{N}\),
że \(n \leq \abs{c}\).
Podzielmy to słowo dowolnie na:
\(w y u z v\), gdzie \(\abs{yz} > 0 \land \abs{yuz} \leq n\).

\begin{itemize}

\item Jeśli \(\$ \in y \lor \$ \in z\), to \(w y^m u z^m v \not \in L_p\) dla \(m \geq 2\), gdyż każde słowo
z \(L_p\) ma dokładnie jedną literę `\$';

\item Dla \(\$ \in w\) zmieniamy prawą stronę słowa, jednakże wielomian to funkcja,
więc nie może tak być, że \(p(d)=e=f, e \neq f\),
zatem \(\exists_{m \in \mathcal{N}} \ w y^m u z^m v \not \in L_p\);

\item Dla \(\$ \in v\) zmieniamy lewą stronę słowa, jednakże w wielomianie stopnia większego niż 1
dla \(w = g \$ h, \exists_{m \in \mathcal{N}} \ p(\abs{g} + m(\abs{y} + \abs{z})) \neq h\),
zatem \(\exists_{m \in \mathcal{N}} \ w y^m u z^m v \not \in L_p\);

\item Dla \(\$ \in u\) zmieniamy środek słowa, jednakże w wielomianie stopnia większego niż 1
dla \(w = g \$ h, \exists_{m \in \mathcal{N}} p(\abs{g} + m \abs{y}) \neq h + m \abs{z}\),
zatem \(\exists_{m \in \mathcal{N}} \ w y^m u z^m v \not \in L_p\).

\end{itemize}

Stąd z lematu o pompowaniu języków bezkontekstowych, sprzeczność.

\end{proof}

Czyli dla takich wielomianów \(p\), \(L_p\) nie jest bezkontekstowy.

\section{}

\subsection{jednej zmiennej}

Rozważmy wielomiany \(p(x,y)\), które są wielomianami defacto jednej zmiennej. Tz., że
\(\forall_{x_1,x_2,y \in \mathcal{Z}} p(x_1,y)=p(x_2,y)\) \ 
lub \(\forall_{y_1,y_2,x \in \mathcal{Z}} p(x,y_1)=p(x,y_2)\). Wtedy przenosimy rozumowanie z podpunktu \ref{sec:a}.

\subsection{liniowe}\label{sec:b:lin}
Rozważmy wielomiany postaci \(p(x,y)=dx + ey + f\), gdzie \(d,e,f \in Z \land d \neq 0, e \neq 0\).
Weźmy język \(L_p\). Pokażę, że istnieje automat ze stosem \(M\),
że \(L_p = L(M)\).

\begin{definition}
    Niech \(M\) będzie automatem ze stosem, że:
    \(M\) na początku na stosie ma \(f\) symboli `+'.
    Zaczyna w stanie \(q_0\).
    Gdy automat przeczyta `a'(/`b)', to zostaje w stanie \(q_0\):
    \begin{itemize}
        \item jeśli przeczyta literę `a'(/`b') to `dokłada' na stos \(d(/e)\) symboli `+' lub `-' w zależności od znaku (+ dla >0, - dla <0):
        \begin{itemize}
            \item jeśli na szczycie jest taki sam symbol jak dodawany, to dodaje \(d(/e)\) razy,
            \item jeśli na szczycie jest inny symbol, to:
            \begin{itemize}
                \item zdejmuje ze stosu \(d(/e)\) razy lub do pustości stosu, wykonując łącznie \(m \leq d(/e)\) operacji,
                \item gdy \(m < d(/e)\) to oznacza, że stos jest pusty, więc dodaje \(d(/e) - m\) razy symbol (ten który pierwotnie miał dodawać),
            \end{itemize}
        \end{itemize}
    \end{itemize}

    Gdy przeczyta pierwsze `\$' przechodzi do stanu \(q_1\).

    W stanie \(q_1\) z każdą literką `a' lub `b' zdejmuje ze stosu jedno `+'.
    Gdy:
    \begin{itemize}
        \item stos jest pusty a słowo jeszcze nie zostało przeczytane,
        \item na górze stosu jest `-',
        \item automat napotka kolejny `\$'
    \end{itemize}
    to przechodzi do śmietnika.

    Automat akceptuje w stanie \(q_1\) wtw. gdy stos jest pusty.
\end{definition}

\begin{proof}
    `\(\subseteq\)'

    Weźmy dowolne słowo \(w \in L_p, w = l \ \$ \ r\).
    Zauważmy, że gdy automat \(M\) będzie czytał \(w\),
    to w momencie przeczytania `\$' na stosie będzie \(d(\#_{a}(l))+e(\#_{b}(l))+f\) symboli `+'.
    Po przeczytaniu `\$' automat przechodzi do stanu \(q_1\).
    W stanie \(q_1\) zdejmuje ze stosu symbol `+' za każdą przeczytaną literę.
    Zatem w momencie przeczytania \(r\) na stosie 0 symboli `+' oraz będziemy w stanie \(q_1\).
    Zatem \(w \in L(M)\).
\end{proof}
\begin{proof}
    `\(\supseteq\)'

    Weźmy dowolne słowo \(w \in L(M), w = l \ \$ \ r\).
    Zauważmy, że zgodnie z definicją automatu,
    na początku na stosie mamy \(f\) symboli `+',
    za każdą literę `a'(/`b') dodajemy do naszego licznika (w formie stosu) \(d/(e)\).
    Akceptujemy tylko słowa, które po symbolu `\$' mają \(d(\#_{a}(l))+e(\#_{b}(l))+f\) liter.
    Zatem \(w \in L_p\).
\end{proof}

Zatem \(L_p = L(M)\).
Stąd \(L_p\) jest bezkontekstowy.

\subsection{o skończonej liczbie punktów o wartości nieujemnej}

Rozważmy wielomiany \(p(x,y)\), że \( Y = \left\{ <x,y>: \ p(x,y) \geq 0 \right\}\) jest zbiorem skończonym.

Zauważmy, że skoro jest to zbiór skończony, to język słów \( X_{<x,y>} = \left\{ w: \ \#_a(w) = x \land \#_b(w) = y \right\} \subset \Sigma^* \) jest skończony.

Również język \(Z_{<x,y>} = \left\{ w: \ \abs{w} = p(x,y) \right\}\) jest językiem skończonym dla każdego \(<x,y> \in Y\), gdyż \(\Sigma\) jest skończony.

Skoro \(Y\) jest zbiorem skończonym, to: 
\(L_p = \bigcup_{<x,y> \in Y} {X_{<x,y>}\ \$ \ Z_{<x,y>}} \) jest skończoną sumą iloczynów języków skończonych, zatem również jest językiem skończonym.

Zatem \(L_p\) jest bezkontekstowy.

Jakie są to wielomiany?
Powinny zbiegac do \(- \infty\) dla \(x \to \infty\) oraz \(y \to \infty\).
Zatem napewno muszą mieć stopień (największą sumę potęg zmiennych przy jednym wyrazie) parzystą (wpp. dla zmiennych dążących do \(-\infty\) by nie zbiegało) oraz współczynnik przy tych wyrazach musi być ujemny.
Dalej trudno jest mi powiedzieć, czy są jakieś inne ograniczenia.

\subsection{sumy wielomianów jednej zmiennej}

Rozważmy wielomiany \(p(x,y) = p_a(x) + p_b(y)\), gdzie \(p_a(x)\) i \(p_b(y)\) są wielomianami jednej zmiennej o współczynnikach całkowitych.
Rozważałem już przypadek, że \(p_a(x)\) i \(p_b(y)\) są liniowe w \ref{sec:b:lin}.

\begin{itemize}
    \item jeśli który kolwiek wielomian jest stopnia większego niż 1 i ma współczynnik przy najwyższej potędze dodatni,
    to \(L_p\) nie jest bezkontekstowy. Wynika to z tego, że wystarczy pominąć drugą literę (poprzez obraz mapujący ją na `$\epsilon$')
    i otrzymujemy język nie bezkontekstwowy, co udowadniałem w \ref{sec:a:remaining}. Języki bezkontekstowe sa zamknięte na obraz,
    zatem \(L_p\) nie jest bezkontekstowy.

    Niżej rozważę przypadki, gdy nie jest to prawda.
    
    \item wpp. oba wielomiany: są conajwyżej liniowe lub mają współczynnik przy najwyższej potędze ujemny.
    Gdy ktoryś ma współczynnik przy najwyższej potędze ujemny, to:
    \begin{itemize}
        \item \(L_p\) jest językiem skończonym, gdy w obydwu wielomianach współczynnik przy najwyższej potędze jest ujemny,
        Skoro \(L_p\) jest językiem skończonym, to również regularny, a zatem bezkontekstowy.
        \item W przeciwnym razie, \(L_p\) jest skończoną sumą jezyków bezkontekstowych,
        zatem również jest językiem bezkontekstowym.
    \end{itemize}
\end{itemize}

\subsection{koniec}

Trudno jest mi powiedzieć czy są jakieś inne przypadki, zatem i trudno o ogólny zwarty opis i dowód, że wpp. nie są bezkontekstowe.


\end{document}