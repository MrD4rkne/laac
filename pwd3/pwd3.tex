\documentclass{article}
\setlength{\headheight}{8pt}

\usepackage[a4paper, margin=0.3in]{geometry}
\usepackage{lmodern}
\usepackage[polish]{babel}
\usepackage[T1]{fontenc}
\usepackage{amsmath, amssymb, amsthm}
\usepackage{hyperref}
\usepackage[shortlabels]{enumitem}
\usepackage{physics}
\usepackage{multicol}
\usepackage{titlesec}
\usepackage{tikz}
\usetikzlibrary{automata, positioning}

% Definicje środowisk do twierdzeń, lematów, dowodów itp.
\newtheorem{theorem}{Twierdzenie}[section]
\newtheorem{lemma}[theorem]{Lemat}
\newtheorem{corollary}[theorem]{Wniosek}
\newtheorem{proposition}[theorem]{Stwierdzenie}
\theoremstyle{definition}
\newtheorem{definition}{Definicja}[section]
\theoremstyle{remark}
\newtheorem*{remark}{Uwaga}
\newtheorem*{example}{Przykład}
\newtheorem*{property}{Własność}

\title{Praca domowa 2}
\author{Marcin Szopa 459531}
\date{\today}

% Remove padding at the top of subsections
\titlespacing*{\subsection}{0pt}{*0}{*0}

% Redefine section numbering to use letters
\renewcommand{\thesection}{\Alph{section}}

% Define a command for singleton sets
\newcommand{\singleton}[1]{\left\{ #1 \right\}}

\begin{document}

\maketitle

\section{niepisząca maszyna Turinga z 1 taśmą roboczą}

Weźmy taką maszynę \(M\).

Zauważmy, że zgodnie z definicją \(P\) jest puste, gdy mamy tylko 1 taśmę roboczą.

Stąd pomińmy pusty zbiór w funkcji przejścia, gdyż nic on nie wnosi. Tz. w dowolnej konfiguracji ten "warunek" jest spełniony: \(\sigma \subseteq, \mathcal{Q} \times A \times \mathcal{Q} \times \left\{ \circlearrowright,  \rightarrow  \right\} \times \left\{ \leftarrow, \circlearrowright,  \rightarrow  \right\}^k\).

Pokażę, że taśma robocza jest bezużyteczna. Wynika to z faktu, że możemy się po niej w sposób nieograniczony poruszać, jednakże nic z niej odczytywać. Stąd tak naprawdę \(M\) to automat.

Zdefiniujmy \(M_A = \left<\Sigma, \right>\)

\end{document}