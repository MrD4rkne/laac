\documentclass{article}
\setlength{\headheight}{8pt}

\usepackage[a4paper, margin=0.3in]{geometry}
\usepackage{lmodern}
\usepackage[polish]{babel}
\usepackage[T1]{fontenc}
\usepackage{amsmath, amssymb, amsthm}
\usepackage{hyperref}
\usepackage[shortlabels]{enumitem}
\usepackage{physics}
\usepackage{multicol}
\usepackage{titlesec}
\usepackage{tikz}
\usetikzlibrary{automata, positioning}

% Definicje środowisk do twierdzeń, lematów, dowodów itp.
\newtheorem{theorem}{Twierdzenie}[section]
\newtheorem{lemma}[theorem]{Lemat}
\newtheorem{corollary}[theorem]{Wniosek}
\newtheorem{proposition}[theorem]{Stwierdzenie}
\theoremstyle{definition}
\newtheorem{definition}{Definicja}[section]
\theoremstyle{remark}
\newtheorem{remark}{Uwaga}[section]
\newtheorem*{example}{Przykład}
\newtheorem*{property}{Własność}

\title{Praca domowa 4}
\author{Marcin Szopa 459531}
\date{\today}

% Remove padding at the top of subsections
\titlespacing*{\subsection}{0pt}{*0}{*0}

% Redefine section numbering to use letters
\renewcommand{\thesection}{\Alph{section}}

% Define a command for singleton sets
\newcommand{\singleton}[1]{\left\{ #1 \right\}}

\begin{document}

\maketitle

\begin{section}{NP}

 Zdefiniujmy maszynę niedeterministyczną, która w czasie wielomianowym
 rozwiązuje podany problem.

 Maszyna na początku niedeterministycznie wybiera podzbiór $W' \subseteq W$ i
 tworzy mieszankę $w$ zgodnie z definicją. Zachowuje przy tym kolejność między
 literami z tego samego słowa, tz. że dla słowa $w = w_1 w_2 \ldots w_n$, jeśli
 litera $w_i$ i $w_j$, $ i < j$ pochodzą z tego samego słowa, to $w_i$ poprzedza
 $w_j$ w słowie $v \in W'$.

 Następnie maszyna sprawdza, czy powstałe słowo jest palindromem. Jeśli nie
 jest, to odrzuca. Wiemy z ćwiczeń, że maszyna umie zrobić to w czasie
 wielomianowym.

 Jeśli słowo jest palindromem, to maszyna symuluje automat $A$ na słowie $w$.
 Jeśli automat $A$ zaakceptuje słowo, to maszyna akceptuje, w przeciwnym razie
 odrzuca. Symulacja również jest w czasie wielomianowym, co robiliśmy na
 ćwiczeniach.

 Zatem w czasie wielomianowym maszyna niedeterministyczna sprawdza, czy istnieje
 taki podzbiór $W' \subseteq W$, że powstała mieszanka $w$ jest palindromem i
 jest akceptowane przez automat $A$.

 Zatem ten problem jest w klasie NP.

\end{section}

\begin{section}{NP trudność}

 By udowodnić NP trudność, pokażemy wielomianową redukcję z problemu 3-CNF-SAT
 dla $n$ zmiennych do problemu zdefiniowanego w zadaniu.

 Zatem mamy formułę w postaci koniunkcji $k$ klauzul $C_1, C_2, \ldots, C_k$,
 gdzie każda klauzula jest alternatywą maksymalnie trzech literałów. Klauzule
 numerujemy liczbami od 1 do $k$ od lewej, a literały od 1 do $n$. Klauzule w
 sumie używają $n$ różnych zmiennych $x_1, x_2, \ldots, x_n$ lub ich negacji.

 \begin{definition}
     $\Sigma = \{ x_1, x_2, \ldots, x_n, \neg x_1, \neg x_2, \ldots, \neg x_n, C_1, C_2, \ldots, C_k, \#, \$ \}$ to alfabet na którym będziemy pracować. Jak widać, jego
     rozmiar jest wielomianowy względem rozmiaru problemu 3-CNF-SAT.
 \end{definition}

 \begin{subsection}{W}
     Skonstruujemy w taki sposób $W$, że dla każdego wartościowania zmiennych $x_1, x_2, \ldots, x_{n}$,
     istnieje taki podzbiór $W' \subseteq W$, że istnieje mieszanka $w$ powstała ze słow z $W'$, która jest palindromem reprezentującym to wartościowanie.

     \begin{definition}
         Dla każdej zmiennej $x_i$ zdefiniujmy dwa ciągi: $T_i$ oraz $F_i$.
         \begin{itemize}
             \item $T_i = (C_j)$, gdzie klauzula $C_j$ zawiera zmienną $x_i$ w postaci pozytywnej, czyli $x_i$. Dodatkowo wewnątrz ciągu zachowujemy kolejność numeryczną klauzul, tz. $\forall_{j, k: j < k} T_j=C_a \land T_k=C_b \implies a < b$.
             \item $F_i = (C_j)$, gdzie klauzula $C_j$ zawiera zmienną $x_i$ w postaci negatywnej, czyli $\neg x_i$.
                   Dodatkowo wewnątrz ciągu zachowujemy kolejność numeryczną klauzul, tz. $\forall_{j, k: j < k} F_j=C_a \land F_k=C_b \implies a < b$.
         \end{itemize}
     \end{definition}

     \begin{definition}
         \label{W_definition}
         $W = \{ T_i \ \$ \ x_i \ \# \ x_i \ T_i^R: i \in \{1, 2, \ldots, n\} \} \cup \{ F_i \ \$ \ \neg x_i \ \# \ \neg x_i \ F_i^R: i \in \{1, 2, \ldots, n\} \}$.
         Również tu rozmiar $W$ jest wielomianowy względem rozmiaru problemu 3-CNF-SAT.
     \end{definition}

     \begin{remark}
         $T_i$ i $F_i$ jako słowa to po prostu ciąg klauzul w postaci reprezentujących je liter według kolejności w ciągu.
     \end{remark}

     \begin{definition}
         \label{evaluation_proof}
         Weźmy dowolne wartościowanie zmiennych $x_1, x_2, \ldots, x_n: b_1, b_2, \ldots, b_n$, gdzie $b_i \in \{0, 1\}$.

         $ W' = \{ T_i \ \$ \ x_i \ \# \ x_i \ T_i^R: i \in \{1, 2, \ldots, n\} \land b_i = 1 \} \cup \{ F_i \ \$ \ \neg x_i \ \# \ \neg x_i \ F_i^R: i \in \{1, 2, \ldots, n\} \land b_i = 0 \}$

         Będziemy oznaczać $Z_i$ jako $T_i$ lub $F_i$ w zależności od wartości $b_i$.
         Oczywiście z konstrukcji powyżej wzieliśmy tylko jeden z ciągów $T_i$ lub
         $F_i$.

         Tak samo oznaczamy $y_i$ jako $x_i$ lub $\neg x_i$ w zależności od wartości
         $b_i$.

         Nietrudno zauważyć, że jedną z mieszanek słów z $W'$ jest:
         \[ w = (Z_1 \ Z_2 \ \ldots \ Z_n)^S \ \$^n \ y_1 \ y_2 \ \ldots \ y_n \ \#^n \ y_n \ y_{n-1} \ \ldots \ y_1 \ \$^n \ (Z_n^R \ Z_{n-1}^R \ \ldots \ Z_1^R)^{RS} \]
         gdzie $S$ to operator sortowania, który sortuje reprezentacje klauzul według
         ich indeksowania. Oczywistym jest, że istnieje mieszanka z tak posortowanymi
         klauzulami, ponieważ nie zaburzamy kolejności klauzul w obrębie słów $T_i$ i
         $F_i$, jedynie w relacji do liter z różnych ciagów. RS to reversal sort, czyli
         sortowanie rewersu, który sortuje klauzule według ich indeksowania, ale w
         odwrotnej kolejności. Oznacza to, że klauzule są posortowane w taki sposób, że
         ostatnia klauzula jest pierwsza, przedostatnia jest druga itd. Tu też nie
         zaburzamy poprawności mieszanki, ponieważ nie zaburzamy kolejności klauzul w
         obrębie słów $T_i^R$ i $F_i^R$, jedynie w relacji do liter z różnych ciagów.

         Oczywistym jest, że powstała mieszanka jest poprawna oraz jest palindromem.
         Zatem dla każdego wartościowania istnieje podzbiór $W' \subseteq W$, który
         tworzy mieszankę $w$ reprezentujących to wartościowanie.

     \end{definition}
 \end{subsection}

 \begin{subsection}{Automat}

     \begin{subsection}{Definicja}

         Idea jest taka, że automat będzie oczekiwał rozwiązania problemu 3-CNF-SAT w
         postaci mieszanki, która będzie:

         \begin{definition}
             Automat $A$ akceptuje przez stan. Zaczyna z pustym stosem:
             \begin{itemize}

                 \item na początku ciągiem liter, $w_0 w_1 \ldots w_m, m \geq k, w_i \in \{ C_1, C_2,
                           \ldots, C_k \} \land w_0 = C_1, w_m = C_k, \forall_{i: 0 < i < m} w_i=C_a \land
                           w_{i+1}=C_b \land 0 \leq b-a \leq 1$. Innymi słowami oczekujemy ciągu liter
                       reprezentujących klauzule posortowane w sposób niemalejący po indeksach
                       klauzul, np.: $C_1 C_1 C_2 C_2 C_3 C_3 C_3 C_4 C_5$ dla $k = 5$.

                       Z powyższej konstrukcji \(W\) wynika to, że do mieszanki wzieliśmy podzbiór
                       słów taki, że obecność klauzuli $C_j$ jest spełniona przez wartościowanie
                       pewnej zmiennej, która jest w mieszance za ciągiem $ \$^n $ w postaci $ x_i $
                       lub $ \neg x_i $, gdzie $ i \in \{1, 2, \ldots, n\} $.

                 \item po ciągu ciąg $C$ to oczekujemy podsłowa $ \$^n $.

                 \item Następnie oczekujemy posortowanego ciągu liter reprezentujących niesprzeczne
                       wartościowania zmiennych. Oczekujemy ich posortowanych po indeksie zmiennych.
                       Każda zmienna może występować tylko w postaci $x_i$ lub $ \neg x_i$. Nie może w
                       obu. Jeśli wystepuje w obu, to automat odrzuca słowo. Zatem oczekujemy ciągu
                       liter w postaci:

                       \[ l_1 l_2 \ldots l_h, h \geq n, l_i \in \{ x_1, x_2, \ldots, x_n, \neg x_1, \neg x_2, \ldots, \neg x_n \} \]
                       gdzie $ l_1 \in \{ x_1, \neg x_1\} \land l_h \in \{ x_n, \neg x_n\} \\ \land
                           \forall_{i: 1 \leq i < h} (((l_i = x_j \land l_{i+1} = x_j) \lor (l_i = \neg
                           x_j \land l_{i+1} = \neg x_j)) \\ \lor \\ (((l_i = x_j \land l_{i+1} = x_{j+1})
                           \lor (l_i = x_j \land l_{i+1} = \neg x_{j+1}) \lor (l_i = \neg x_j \land
                           l_{i+1} = x_{j+1}) \lor (l_i = \neg x_j \land l_{i+1} = \neg x_{j+1})))) $.

                       Przykładem poprawnego ciągu jest $x_1 x_1 x_2 \neg x_2 x_3 \neg x_3$ dla
                       wartościowania $x_1 = 1, x_2 = 0, x_3 = 0$ oraz $n = 3$. Jednakże ciąg $x_1 x_2
                           \neg x_2 x_3 \neg x_3$ nie jest poprawny, ponieważ występuje w nim zarówno
                       $x_2$ jak i $\neg x_2$, co nie jest poprawnym wartościowaniem zmiennych.

                 \item Następnie oczekujemy ciągu $ \#^n$. Gdy wczytamy taki ciąg, to automat
                       przechodzi do stanu akceptującego i wszystkie następne przejścia nie
                       wyprowadzają z tego stanu.

             \end{itemize}

             Wszystkie nieoczekiwane litery prowadzą do śmietnika.

             Zauważmy, że obecność separatorów $ \$^n $ oraz $ \#^n $ w mieszance jest
             konieczna, aby wymusić rozdzielenie klauzul i wartościowań zmiennych oraz część
             faktyczną od rewersu, który jest tylko dla palindromowatości. Definicja
             mieszanki wymaga, aby litery w obrębie słowa z którego pochodzą były
             uporządkowane jak w słowie, zatem wymuszenie spójnego podciągu separatorów
             sprawia, że informacje się nie mieszają.

         \end{definition}

         \begin{remark}
             Zdefiniowany automat jest rozmiaru wielomianowego względem rozmiaru problemu
             3-CNF-SAT.
         \end{remark}

         \begin{remark}
             Zauważmy jako ciekawostkę, że nie korzystamy ze stosu. Akceptujemy przez stan.
         \end{remark}

     \end{subsection}

     \begin{subsection}{Automat akceptuje tylko poprawne wartościowania spełniające formułę}

         Udowodnimy, że ten automat akceptuje tylko mieszanki będące poprawnymi
         wartościowaniami zmiennych $x_1, x_2, \ldots, x_n$ w rozważanym problemie
         3-CNF-SAT spełniającym formułę.

         \begin{proof}
             Załóżmy, że istnieje mieszanka $w$, która jest akceptowana przez automat
             $A$.

             Z definicji automatu najpierw musi być ciąg liter reprezentujących posortowany
             po indeksach ciąg wszystkich klauzul, każda występuje conajmniej raz. Zatem ta
             część wartościowania jest poprawna.

             Następnie musi wystąpić ciąg $ \$^n $, który oddziela część z klauzulami od
             części z wartościowaniami zmiennych. Zapewnia nam to, że słowo jest mieszanką
             słów z pewnego podzbioru $W' \subseteq W$, ŻE $ \abs{W'} = n $. Gdyby tak nie
             było, to liczba wystąpień separatora $ \$ $ byłaby inna niż $n$, więc automat
             by odrzucił takie słowo.

             Następnie musi wystąpić ciąg liter reprezentujących wartościowania zmiennych
             $x_1, x_2, \ldots, x_n$. Z definicji wartościowania zmiennych w problemie
             3-CNF-SAT, każda zmienna może wystąpić tylko w postaci $x_i$ lub $\neg x_i$.
             Moga być powtórzenia, ale nie mogą wystąpić obie formy tej samej zmiennej.
             Zatem część z wartościowaniami zmiennych jest poprawna. Czyli wartościowanie
             jest poprawne w sensie braku sprzeczności. Dodatkowo zapewniamy tu też, że $W'$
             zawiera słowo dla każdej zmiennej (patrz: \ref{W_definition}). Gdyby nie
             zawierało, to opisywany warunek nie był by spełniony, bo by brakowało pewnego
             $x_i$ lub $\neg x_i$.

             Następnie musi wystąpić ciąg $ \#^n $, który oddziela część z wartościowaniami
             zmiennych od części z rewersem. Zapewnia ono, że litery potrzebne tylko i
             wyłącznie do rewersu nie `przeciekły' do części z wartościowaniami zmiennych.

             Zatem istnieje wartościowanie zmiennych $x_1, x_2, \ldots, x_n$, które jest
             reprezentowane przez to słowo $w$.

             Skoro istnieje takie wartościowanie, to spełnia ono formułę $C_1 \land C_2
                 \land \ldots \land C_k$. W przeciwnym razie, gdyby nie spełniało, to nie
             mogłoby powstać z podzbioru $W' \subseteq W$. Z

             Zatem automat $A$ akceptuje tylko mieszanki będące poprawnymi wartościowaniami
             zmiennych $x_1, x_2, \ldots, x_n$ w rozważanym problemie 3-CNF-SAT spełniającym
             tę formułę.
         \end{proof}

     \end{subsection}

     \begin{subsection}{Automat akceptuje wszystkie poprawne wartościowania spełniające formułę}
         Pokażmy jeszcze, że automat akceptuje wszystkie mieszanki będące (poprawnymi)
         wartościowaniami zmiennych $x_1, x_2, \ldots, x_n$ w rozważanym problemie
         3-CNF-SAT spełniającym tę formułę.

         \begin{proof}
             Weźmy dowolne wartościowanie zmiennych $x_1, x_2, \ldots, x_n$ w rozważanym problemie 3-CNF-SAT.
             Wiemy, że istnieje taki podzbiór $W' \subseteq W$, który tworzy mieszankę $w$ reprezentującą to wartościowanie (patrz: \ref{evaluation_proof}).

             Na początku mieszanka $w$ będzie zawierała ciąg liter reprezentujących
             posortowany po indeksach ciąg wszystkich klauzul, każda występuje conajmniej
             raz. Zatem ta faza automatu zaakceptuje to słowo.

             Następnie musi wystąpić ciąg $ \$^n $, który oddziela część z klauzulami od
             części z wartościowaniami zmiennych. Z definicji reprezentacji wartościowania
             tak jest.

             Teraz musi wystąpić ciąg liter reprezentujących wartościowania zmiennych $x_1,
                 x_2, \ldots, x_n$. Z definicji reprezentacji wartościowania tak jest. Dodatkowo
             zapewniamy tu też, że $W'$ zawiera słowo dla każdej zmiennej (patrz:
             \ref{evaluation_proof}).

             Następnie musi wystąpić ciąg $ \#^n $, który oddziela część z wartościowaniami
             zmiennych od części z rewersem. Znów, wiemy, że tak jest z definicji.

             Automat przechodzi do stanu akceptującego i wszystkie następne przejścia nie
             wyprowadzają z tego stanu. Zatem automat akceptuje to słowo.

         \end{proof}
     \end{subsection}
 \end{subsection}

 \begin{subsection}{Podsumowanie redukcji}

     Pokazaliśmy konstrukcję reprezentacji wartościowań oraz automatu, który
     akceptuje tylko poprawne wartościowania.

 \end{subsection}

 \begin{subsection}{Poprawność redukcji}

     Udowodnijmy, że redukcja jest poprawna.

     \begin{proof}

         Weźmy dowolny problem 3-CNF-SAT z formułą $C_1 \land C_2 \land \ldots \land
             C_k$ używającą zmiennych $x_1, x_2, \ldots, x_n$. Zdefiniowaliśmy reprezentację
         $W$ oraz automat $A$.

         Załóżmy, że istnieje wartościowanie zmiennych $x_1, x_2, \ldots, x_n$, które
         spełnia formułę. Z definicji reprezentacji $W$ oraz konstrukcji automatu $A$
         istnieje podzbiór $W' \subseteq W$, który tworzy mieszankę $w$ reprezentującą
         to wartościowanie. Ponieważ wartościowanie jest poprawne, to mieszanka $w$ jest
         palindromem i jest akceptowana przez automat $A$. Zatem odpowiedź na
         zredukowany problem bedzie "tak".

         Załóżmy teraz, że nie istnieje takie wartościowanie. Wtedy z konstrukcji
         reprezentacji $W$ oraz automatu $A$ nie istnieje taki podzbiór $W' \subseteq
             W$, który tworzy mieszankę $w$ reprezentującą to wartościowanie. Wynika to z
         tego, że niezależnie od tego, jakie wartościowanie weźmiemy, to po ciągu $ \$^n
         $ znajdziemy pewne zaprzeczenie, tz. pewną parę liter $ x_i $ i $ \neg x_i $,
         które nie mogą wystąpić w tej samej mieszance. Zatem nie istnieje taka
         mieszanka $w$, która jest palindromem i jest akceptowana przez automat $A$.
         Zatem odpowiedź na zredukowany problem będzie "nie".

         Czyli zachowujemy `wierność` redukcji, czyli dla każdego problemu 3-CNF-SAT
         istnieje taki podzbiór $W' \subseteq W$, że mieszanka $w$ jest palindromem i
         jest akceptowana przez automat $A$ wtedy i tylko wtedy, gdy istnieje takie
         wartościowanie zmiennych $x_1, x_2, \ldots, x_n$, które spełnia formułę $C_1
             \land C_2 \land \ldots \land C_k$.

         Na każdym kroku naszej redukcji zachowujemy rozmiar wielomianowy względem
         rozmiaru problemu 3-CNF-SAT. Zatem redukcja jest wielomianowa. Stąd redukcja
         jest poprawna.

     \end{proof}

 \end{subsection}

 Skoro redukcja jest poprawna, to problem NP-trudny redukuje się do problemu z
 treści zadania w czasie wielomianowym.

\end{section}

\section {Podsumowanie}

Pokazaliśmy, że problem jest w klasie NP oraz jest NP-trudny (bo pewien problem
NP-trudny redukuje się do niego, a więc wszystkie problemy w NP redukują się do
niego).

Zatem problem jest NP-zupełny.

\end{document}