\documentclass{article}
\setlength{\headheight}{8pt}

\usepackage[a4paper, margin=0.3in]{geometry}
\usepackage{lmodern}
\usepackage[polish]{babel}
\usepackage[T1]{fontenc}
\usepackage{amsmath, amssymb, amsthm}
\usepackage{hyperref}
\usepackage[shortlabels]{enumitem}
\usepackage{physics}
\usepackage{multicol}
\usepackage{titlesec}
\usepackage{tikz}
\usetikzlibrary{automata, positioning}

% Definicje środowisk do twierdzeń, lematów, dowodów itp.
\newtheorem{theorem}{Twierdzenie}[section]
\newtheorem{lemma}[theorem]{Lemat}
\newtheorem{corollary}[theorem]{Wniosek}
\newtheorem{proposition}[theorem]{Stwierdzenie}
\theoremstyle{definition}
\newtheorem{definition}{Definicja}[section]
\theoremstyle{remark}
\newtheorem{remark}{Uwaga}[section]
\newtheorem*{example}{Przykład}
\newtheorem*{property}{Własność}

\title{Praca domowa 4}
\author{Marcin Szopa 459531}
\date{\today}

% Remove padding at the top of subsections
\titlespacing*{\subsection}{0pt}{*0}{*0}

% Redefine section numbering to use letters
\renewcommand{\thesection}{\Alph{section}}

% Define a command for singleton sets
\newcommand{\singleton}[1]{\left\{ #1 \right\}}

\begin{document}

\maketitle

\begin{section}{NP}

Zdefiniujmy maszynę niedeterministyczną, która w czasie wielomianowym rozwiązuje podany problem.

Maszyna na początku niedeterministycznie wybiera podzbiór $W' \subseteq W$ i tworzy mieszankę $w$ zgodnie z definicją.
Zachowuje przy tym kolejność między literami z tego samego słowa, tz. że dla słowa $W = w_1 w_2 \ldots w_n$, jeśli litera $w_i$ i $w_j$, $ i < j$ pochodzą z tego samego słowa, to
$w_i$ poprzedza $w_j$ w słowie $v \in W'$.

Następnie maszyna sprawdza, czy powstałe słowo jest palindromem. Jeśli nie jest, to odrzuca. Wiemy z ćwiczeń, że maszyna umie zrobić to w czasie wielomianowym.

Jeśli słowo jest palindromem, to maszyna symuluje automat $A$ na słowie $w$. Jeśli automat $A$ zaakceptuje słowo, to maszyna akceptuje, w przeciwnym razie odrzuca.
Symulacja również jest w czasie wielomianowym, co robiliśmy na ćwiczeniach.

Zatem w czasie wielomianowym maszyna niedeterministyczna sprawdza, czy istnieje taki podzbiór $W' \subseteq W$,
że powstała mieszanka $w$ jest palindromem i jest akceptowane przez automat $A$.

Zatem ten problem jest w klasie NP.

\end{section}

\begin{section}{NP trudność}

By udowodnić NP trudność, pokażemy wielomianową redukcję z problemu 3-CNF-SAT dla $n$ zmiennych do problemu zdefiniowanego w zadaniu.

Zatem mamy formułę w postaci koniunkcji $k$ klauzul $C_1, C_2, \ldots, C_k$, gdzie każda klauzula jest alternatywą maksymalnie trzech literałów.
Klauzule numerujemy liczbami od 1 do $k$ od lewej, a literały od 1 do $n$.
Klauzule w sumie używają $n$ zmiennych $x_1, x_2, \ldots, x_n$ lub ich negacji.

\begin{definition}
    $\Sigma = \{ x_1, x_2, \ldots, x_n, \neg x_1, \neg x_2, \ldots, \neg x_n, C_1, C_2, \ldots, C_k, \#, \$ \}$ to alfabet na którym będziemy pracować. Jak widać, jego
    rozmiar jest wielomianowy względem rozmiaru problemu 3-CNF-SAT.
\end{definition}

\begin{subsection}{W}
    Skonstruujemy w taki sposób $W$, że dla każdego wartościowania zmiennych $x_1, x_2, \ldots, x_{n}$,
     istnieje taki podzbiór $W' \subseteq W$, że istnieje mieszanka $w$ powstała ze słow z $W'$, która jest palindromem reprezentującym to wartościowanie.

     \begin{definition}
        Dla każdej zmiennej $x_i$ zdefiniujmy dwa ciągi: $T_i$ oraz $F_i$.
        \begin{itemize}
            \item $T_i = (C_j)$, gdzie klauzula $C_j$ zawiera zmienną $x_i$ w postaci pozytywnej, czyli $x_i$. Dodatkowo wewnątrz ciągu zachowujemy kolejność numeryczną klauzul, tz. $\forall_{j, k: j < k} T_j=C_a \land T_k=C_b \implies a < b$.
            \item $F_i = (C_j)$, gdzie klauzula $C_j$ zawiera zmienną $x_i$ w postaci negatywnej, czyli $\neg x_i$.
            Dodatkowo wewnątrz ciągu zachowujemy kolejność numeryczną klauzul, tz. $\forall_{j, k: j < k} F_j=C_a \land F_k=C_b \implies a < b$.
        \end{itemize}
     \end{definition}

     \begin{definition}
        $W = \{ T_i \ \$ \ x_i \ \# \ x_i \ T_i^R: i \in \{1, 2, \ldots, n\} \} \cup \{ F_i \ \$ \ \neg x_i \ \# \ \neg x_i \ F_i^R: i \in \{1, 2, \ldots, n\} \}$.
        Również tu rozmiar $W$ jest wielomianowy względem rozmiaru problemu 3-CNF-SAT.
     \end{definition}

     Idea jest taka, że automat będzie oczekiwał rozwiązania problemu 3-CNF-SAT w postaci mieszanki, która będzie:
     na początku ciągiem liter, $w_0, w_1, \ldots, w_m, m \geq k, w_i \in \{ C_1, C_2, \ldots, C_k \} \land w_0 = C_1, w_m = C_k, \forall_{i: 0 < i < m} w_i=C_a \land w_{i+1}=C_b \land 0 \leq b-a \leq 1$.
     Innymi słowami oczekujemy ciągu liter reprezentujących klauzule posortowane w sposób niemalejący po indeksach klauzul.

     Z powyższej konstrukcji \(W\) wynika to, że do mieszanki wzieliśmy podzbiór słów taki, że obecność klauzuli $C_j$ jest spełniona przez wartościowanie pewnej zmiennej, która jest w mieszance
     za ciągiem $ \$^n $ w postaci $ x_i $ lub $ \neg x_i $, gdzie $ i \in \{1, 2, \ldots, n\} $.

     Gdy wczytamy taki ciąg $C$ to oczekujemy podsłowa $ \$^n $.

     Następnie oczekujemy posortowanego ciągu liter reprezentujących zabronione wartościowania zmiennych.
     Oczekujemy ich posortowanych po indeksie zmiennych. Każda zmienna może występować tylko w postaci $x_i$ lub $ \neg x_i$. Nie może w obu. Jeśli wystepuje w obu,
     to automat odrzuca słowo.
    Zatem oczekujemy ciągu liter w postaci:

    \[ l_1 l_2 \ldots l_h, h \geq n, l_i \in \{ x_1, x_2, \ldots, x_n, \neg x_1, \neg x_2, \ldots, \neg x_n \} \]
    gdzie
    $ l_1 \in \{ x_1, \neg x_1\} \land l_h \in \{ x_n, \neg x_n\} \\
    \land \forall_{i: 1 \leq i < h}
    (((l_i = x_j \land l_{i+1} = x_j) \lor (l_i = \neg x_j \land l_{i+1} = \neg x_j)) \\
    \lor \\
    (((l_i = x_j \land l_{i+1} = x_{j+1}) \lor (l_i = x_j \land l_{i+1} = \neg x_{j+1}) \lor (l_i = \neg x_j \land l_{i+1} = x_{j+1}) \lor (l_i = \neg x_j \land l_{i+1} = \neg x_{j+1}))))
    $.

    Następnie oczekujemy ciągu $ \#^n$. Gdy wczytamy taki ciąg, to automat przechodzi do stanu akceptującego i wszystkie następne przejścia nie wyprowadzają z tego stanu.
    
    Wszystkie nieoczekiwane litery prowadzą do śmietnika.
\end{subsection} 

\end{section}

\end{document}