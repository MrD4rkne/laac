\documentclass{article}
\setlength{\headheight}{8pt}

\usepackage[a4paper, margin=0.3in]{geometry}
\usepackage{lmodern}
\usepackage[polish]{babel}
\usepackage[T1]{fontenc}
\usepackage{amsmath, amssymb, amsthm}
\usepackage{hyperref}
\usepackage[shortlabels]{enumitem}
\usepackage{physics}
\usepackage{multicol}
\usepackage{titlesec}
\usepackage{tikz}
\usetikzlibrary{automata, positioning}

% Definicje środowisk do twierdzeń, lematów, dowodów itp.
\newtheorem{theorem}{Twierdzenie}[section]
\newtheorem{lemma}[theorem]{Lemat}
\newtheorem{corollary}[theorem]{Wniosek}
\newtheorem{proposition}[theorem]{Stwierdzenie}
\theoremstyle{definition}
\newtheorem{definition}{Definicja}[section]
\theoremstyle{remark}
\newtheorem{remark}{Uwaga}[section]
\newtheorem*{example}{Przykład}
\newtheorem*{property}{Własność}

\title{Praca domowa 4}
\author{Marcin Szopa 459531}
\date{\today}

% Remove padding at the top of subsections
\titlespacing*{\subsection}{0pt}{*0}{*0}

% Redefine section numbering to use letters
\renewcommand{\thesection}{\Alph{section}}

% Define a command for singleton sets
\newcommand{\singleton}[1]{\left\{ #1 \right\}}

\begin{document}

\maketitle

\begin{section}{NP}

Zdefiniujmy maszynę niedeterministyczną, która w czasie wielomianowym rozwiązuje podany problem.

Maszyna na początku niedeterministycznie wybiera podzbiór $W' \subseteq W$ i tworzy mieszankę $w$ zgodnie z definicją.
Zachowuje przy tym kolejność między literami z tego samego słowa, tz. że dla słowa $W = w_1 w_2 \ldots w_n$, jeśli litera $w_i$ i $w_j$, $ i < j$ pochodzą z tego samego słowa, to
$w_i$ poprzedza $w_j$ w słowie $v \in W'$.

Następnie maszyna sprawdza, czy powstałe słowo jest palindromem. Jeśli nie jest, to odrzuca. Wiemy z ćwiczeń, że maszyna umie zrobić to w czasie wielomianowym.

Jeśli słowo jest palindromem, to maszyna symuluje automat $A$ na słowie $w$. Jeśli automat $A$ zaakceptuje słowo, to maszyna akceptuje, w przeciwnym razie odrzuca.
Symulacja również jest w czasie wielomianowym, co robiliśmy na ćwiczeniach.

Zatem w czasie wielomianowym maszyna niedeterministyczna sprawdza, czy istnieje taki podzbiór $W' \subseteq W$,
że powstała mieszanka $w$ jest palindromem i jest akceptowane przez automat $A$.

Zatem ten problem jest w klasie NP.

\end{section}

\begin{section}{NP trudność}

By udowodnić NP trudność, pokażemy wielomianową redukcję z problemu 3-CNF-SAT dla $n$ zmiennych do problemu zdefiniowanego w zadaniu.

BSO. zakładamy, że $n$ jest parzyste. Jeśli nie jest, wystarczy dodać zmienną $x_{n+1}$ oraz klauzulę $(x_{n+1} \lor \neg x_{n+1})$.

\begin{subsection}{W}
Zdefiniujmy takie $W$, aby dla każdego wartościowania zmiennych $x_1, x_2, \ldots, x_n$ istniał taki podzbiór $W' \subseteq W$,
 że powstała mieszanka $w$ jest palindromem reprezentującym to wartościowanie.

Wartościowaniem $n$ zmiennych $x_1=b_1, x_2=b_2, \ldots, x_n=b_n, b_i \in \{0,1\}$ będzie słowo $w = b_1 b_2 \ldots b_n \# b_n, b_{n-1}, \ldots, b_1$.

\begin{definition}
Niech $d_{i, b} = b^i, 0 \leq i \leq n \land b \in \{0,1\}$.
Wtedy $W = \{ d_{i, b} + \# d_{i, b} \#: i = 0, 1, \ldots, n;  b \in \{0,1\} \}$.

\begin{remark}
$\abs{W} = 4(n+1)$.
\end{remark}

\begin{remark}
Każde słowo $d_{i,b}$ jest palindromem, zatem każde postaci $ \# d_{i,b} \#$ również jest palindromem.
Stąd wszystkie słowa w $W$ są palindromami.
\end{remark}

\end{definition}

Pokażmy, że dla każdego wartościowania istnieje taki podzbiór $W' \subseteq W$, że można stworzyć mieszankę $w$, która jest palindromem i reprezentuje to wartościowanie.

Dla ciągu zmiennych $x_1, x_2, \ldots, x_n$ z wartościowaniem $b_1, b_2, \ldots, b_n$ zdefiniujmy słowo $v$ jako:
\[
v = \# \# b_1 b_2 \ldots b_n b_n b_{n-1} \ldots b_1 \# \#
\]
\begin{proof}
Weźmy dowolne wartościowanie $b_1, b_2, \ldots, b_n$. Ma ono $i$ zmiennych równych $1$ i $n-i$ zmiennych równych $0$.
Zatem aby otrzymać reprezentację tego wartościowania musimy wybrać takie \(W' \subseteq W\), że w sumie słowa w \(W'\) będą miały:
\begin{itemize}
    \item $2i$ liter równych 1
    \item $2n - 2i$ liter równych 0
    \item 4 litery równe $\#$
\end{itemize}

Zauważmy, że wystarczy wziąć $W' = \{ d_{i,1}, \#d_{i,1}\#, d_{n-i, 0}, \# d_{n-i, 0} \# \}$, gdzie $d_i$ jest jak wyżej zdefiniowane.

Pokażmy, że da się prawidłowo utworzyc mieszankę reprezentującą wartościowanie.

\begin{center}
    $d_{i,1} \ d_{n-i, 0} \ \#d_{i,1}\# \ \# d_{n-i, 0} \#$
    
    $ \# d_{i,1} \ d_{n-i, 0} \ d_{i,1}\# \ \# d_{n-i, 0} \#$

    $ \# \# d_{i,1} \ d_{n-i, 0} \ d_{i,1}\# \ d_{n-i, 0} \#$

    $ \# \# d_{i,1} \ d_{n-i, 0} \ d_{i,1} \ d_{n-i, 0} \#\#$

\end{center}

W $d_{i,1}$ mamy $i$ liter równych 1, w $d_{n-i,0}$ mamy $n-i$ liter równych 0.
Zatem:
\begin{center}
    $ \# \# d_{i,1} \ d_{n-i, 0} \ d_{i,1} \ d_{n-i, 0} \#\#$

    $\# \# b_1 \ b_2 \ \ldots \ b_n \ d_{i,1} \ d_{n-i, 0} \#\#$

    $\# \# b_1 \ b_2 \ \ldots \ b_n \ b_n \ b_{n-1} \ \ldots \ b_1 \#\#$
\end{center}

\end{proof}

Co udowadnia, że dla każdego wartościowania istnieje taki podzbiór $W' \subseteq W$, że istnieje mieszanka $w$ słów z tego zbioru, która jest palindromem i reprezentuje to wartościowanie.

\end{subsection}

\begin{subsection}{A}
Zdefiniujmy automat $A$. W będzie akceptował tylko takie $w$, które są poprawnymi wartościowaniami zmiennych $x_1, x_2, \ldots, x_n$ 
oraz spełniają formułę $\phi$.

Automat na początku na stosie będzie miał ciąg symboli reprezentujący formułę $\phi$.
Mianowicie stos do dosłownie formuła $\phi$ w postaci koniunkcji klauzul, gdzie każda klauzula jest reprezentowana jako ciąg symboli, np:

$ \phi = (x_1 \lor \neg x_2) \land (x_2 \lor x_3)$ będzie na stosie reprezentowane jako:


gdzie od lewej jest początek stosu, czyli od prawej jego szczyt.
\end{subsection}

\end{section}

\end{document}