\documentclass{article}
\setlength{\headheight}{8pt}

\usepackage{anyfontsize}

% \renewcommand\normalsize{\fontsize{14}{20}\selectfont}
% \renewcommand\small{\fontsize{16}{16}\selectfont}
% \renewcommand\footnotesize{\fontsize{14}{14}\selectfont}
% \renewcommand\scriptsize{\fontsize{12}{12}\selectfont}
% \renewcommand\tiny{\fontsize{10}{10}\selectfont}
% \renewcommand\large{\fontsize{20}{20}\selectfont}
% \renewcommand\Large{\fontsize{22}{22}\selectfont}
% \renewcommand\LARGE{\fontsize{24}{24}\selectfont}
% \renewcommand\huge{\fontsize{26}{26}\selectfont}
% \renewcommand\Huge{\fontsize{28}{28}\selectfont}

\usepackage[
    paperwidth=5.04in,
    paperheight=3.78in,
    margin=0.3in
]{geometry}
\usepackage{lmodern}
\usepackage[polish]{babel}
\usepackage[T1]{fontenc}
\usepackage{amsmath, amssymb, amsthm}
\usepackage{hyperref}
\usepackage[shortlabels]{enumitem}
\usepackage{physics}
\usepackage{multicol}
\usepackage{titlesec}
\usepackage{tikz}
\usetikzlibrary{automata, positioning}

% Definicje środowisk do twierdzeń, lematów, dowodów itp.
\newtheorem{theorem}{Twierdzenie}[section]
\newtheorem{lemma}[theorem]{Lemat}
\newtheorem{corollary}[theorem]{Wniosek}
\newtheorem{proposition}[theorem]{Stwierdzenie}
\theoremstyle{definition}
\newtheorem{definition}{Definicja}[section]
\theoremstyle{remark}
\newtheorem{remark}{Uwaga}[section]
\newtheorem*{example}{Przykład}
\newtheorem*{property}{Własność}

\title{Zadania z gwiazdką 3}
\author{Marcin Szopa 459531}
\date{\today}

% Remove padding at the top of subsections
\titlespacing*{\subsection}{0pt}{*0}{*0}

% Redefine section numbering to use letters
\renewcommand{\thesection}{\Alph{section}}

% Define a command for singleton sets
\newcommand{\singleton}[1]{\left\{ #1 \right\}}

\begin{document}

\begin{table}[ht]
    \centering
    \begin{tabular}{|l|l|l|l|}
        \hline
        \textbf{Algorytm}       & \textbf{Zastosowanie}    & \textbf{Typ}             & \textbf{Rekomendowana długość klucza}  \\
        \hline
        AES                     & Szyfrowanie              & Symetryczny blokowy      & 128, 192, 256 bitów                    \\
        3DES                    & Szyfrowanie              & Symetryczny blokowy      & 112, 168 bitów (wycofywany)            \\
        Blowfish                & Szyfrowanie              & Symetryczny blokowy      & do 448 bitów                           \\
        IDEA                    & Szyfrowanie              & Symetryczny blokowy      & 128 bitów                              \\
        CAST-128/256            & Szyfrowanie              & Symetryczny blokowy      & 128/256 bitów                          \\
        RC4                     & Szyfrowanie strumieniowe & Symetryczny strumieniowy & do 256 bitów (niezalecany)             \\
        RC5/RC6                 & Szyfrowanie              & Symetryczny blokowy      & do 2040 bitów (RC5), 128/192/256 (RC6) \\
        ChaCha20                & Szyfrowanie strumieniowe & Symetryczny strumieniowy & 256 bitów                              \\
        \hline
        RSA                     & Szyfrowanie, podpisy     & Asymetryczny             & min. 2048 bitów (zalecane 3072+)       \\
        ElGamal                 & Szyfrowanie, podpisy     & Asymetryczny             & min. 2048 bitów                        \\
        DSA                     & Podpisy cyfrowe          & Asymetryczny             & min. 2048 bitów                        \\
        Diffie-Hellman          & Wymiana kluczy           & Asymetryczny             & min. 2048 bitów                        \\
        ECDSA                   & Podpisy cyfrowe          & Asymetryczny (ECC)       & 256, 384, 521 bitów                    \\
        ECDH                    & Wymiana kluczy           & Asymetryczny (ECC)       & 256, 384, 521 bitów                    \\
        Ed25519                 & Podpisy cyfrowe          & Asymetryczny (ECC)       & 256 bitów                              \\
        Curve25519              & Wymiana kluczy           & Asymetryczny (ECC)       & 256 bitów                              \\
        \hline
        SHA-2 (SHA-256/384/512) & Funkcja skrótu           & Hash                     & 256/384/512 bitów (długość skrótu)     \\
        SHA-3                   & Funkcja skrótu           & Hash                     & 224/256/384/512 bitów (długość skrótu) \\
        RIPEMD-160              & Funkcja skrótu           & Hash                     & 160 bitów (długość skrótu)             \\
        HMAC                    & MAC                      & Hash                     & zależna od funkcji hash (np. 256)      \\
        \hline
    \end{tabular}
    \caption{Nowoczesne algorytmy kryptograficzne, ich przeznaczenie i rekomendowana długość klucza}
\end{table}

\end{document}