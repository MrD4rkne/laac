\documentclass{article}
\setlength{\headheight}{8pt}

\usepackage{anyfontsize}

\renewcommand\normalsize{\fontsize{12}{24}\selectfont}
\renewcommand\small{\fontsize{14}{14}\selectfont}
\renewcommand\footnotesize{\fontsize{12}{12}\selectfont}
\renewcommand\scriptsize{\fontsize{10}{10}\selectfont}
\renewcommand\tiny{\fontsize{8}{8}\selectfont}
\renewcommand\large{\fontsize{18}{18}\selectfont}
\renewcommand\Large{\fontsize{20}{20}\selectfont}
\renewcommand\LARGE{\fontsize{22}{22}\selectfont}
\renewcommand\huge{\fontsize{24}{24}\selectfont}
\renewcommand\Huge{\fontsize{26}{26}\selectfont}

\usepackage[
    paperwidth=5.04in,
    paperheight=3.78in,
    margin=0.3in
]{geometry}
\usepackage{lmodern}
\usepackage[polish]{babel}
\usepackage[T1]{fontenc}
\usepackage{amsmath, amssymb, amsthm}
\usepackage{hyperref}
\usepackage[shortlabels]{enumitem}
\usepackage{physics}
\usepackage{multicol}
\usepackage{titlesec}
\usepackage{tikz}
\usetikzlibrary{automata, positioning}

% Definicje środowisk do twierdzeń, lematów, dowodów itp.
\newtheorem{theorem}{Twierdzenie}[section]
\newtheorem{lemma}[theorem]{Lemat}
\newtheorem{corollary}[theorem]{Wniosek}
\newtheorem{proposition}[theorem]{Stwierdzenie}
\theoremstyle{definition}
\newtheorem{definition}{Definicja}[section]
\theoremstyle{remark}
\newtheorem{remark}{Uwaga}[section]
\newtheorem*{example}{Przykład}
\newtheorem*{property}{Własność}

\title{Zadania z gwiazdką 3}
\author{Marcin Szopa 459531}
\date{\today}

% Remove padding at the top of subsections
\titlespacing*{\subsection}{0pt}{*0}{*0}

% Redefine section numbering to use letters
\renewcommand{\thesection}{\Alph{section}}

% Define a command for singleton sets
\newcommand{\singleton}[1]{\left\{ #1 \right\}}

\begin{document}

\section{NP-zupełne}

\begin{itemize}
    \item Cykl/Ścieżka Hamiltona \\
    problem polegający na znalezieniu cyklu/ścieżki w grafie, który odwiedza każdy wierzchołek dokładnie raz.
    
    \item Problem plecakowy (szczególny przypadek) \\
    zbiór liczb $A=\{n_1,n_2,\ldots,n_k\}$ i liczba $m$ binarnie, pytanie czy $\exists_{A' \subseteq A} \sum_{b \in A'}= m$.
    
    \item Problem SAT \\
    formuła zdaniowa $\phi$, pytanie, czy $\phi$ jest spełnialna

    \item PCP z ograniczeniem \\
    ciąg par słów $(w_1, v_1), (w_2, v_2), \ldots, (w_n, n_n)$ i liczba $k$,
     pytanie czy istnieje takie $1 \leq m \leq k$ ciąg indeksów $i_1, i_2, \ldots, i_m$ że $ w_{i_1} \ldots w_{i_m} = v_{i_1} \ldots v_{i_m}$.
    
    \item Problem 3-CNF-SAT \\
    SAT, ale w każdej klauzuli są <= 3 literały (równowaznie dokładnie 3 w każdek kluazuli).
    
    \item Trójkolorowanie grafu \\
    dany graf $G$, pytanie, czy istnieje taki przydział kolorów $\{1,2,3\}$ do wierzchołków $G$, że żadne dwa sąsiadujące wierzchołki nie mają tego samego koloru.
    
    \item Programowanie całkowitoliczbowe \\
    układ równań liniowych $U$ o całkowitych współczynnikach. Pytanie, czy $U$ ma nieujemne całkowite rozwiązanie.
    
    \item Nierównośc wyrażeń regularnych bez `*' \\
    dwa wyrażenia regularne $R$ i $S$. Pytanie, czy $L(R) \neq L(S)$.
    
    \item Problem stopu po n krokach \\
    niedeterministyczna maszyna Turinga $M$, słowo wejściowe $w$ i $n$ unarnie. Pytanie czy $M$ akceptuje $w$ conajwyżej $n$ krokach?
\end{itemize}

\newpage

\section {Nierozstrzygalne}
\subsection{Częściowo rozstrzygalne}
\begin{itemize}
    \item $\text{Halt}_w$ \\ 
    pytanie, czy dana maszyna Turinga $M$ zatrzyma się na słowie wejściowym $w$. 
    
    \begin{itemize}
    \item szczególny przypadek, $\text{Halt}_{\epsilon}$ - pytanie, czy maszyna Turinga $M$ zatrzyma się na pustym słowie wejściowym.
    \end{itemize}

    \newpage
    
    \item Problem regularności języków bezkontekstowych \\ 
    pytanie, czy dany język bezkontekstowy $L$ jest regularny.
    
    \item Problem jednoznaczności języków bezkontekstowych \\
    pytanie, czy dany język bezkontekstowy $L$ jest jednoznaczy, tz. dla każdego słowa istnieje tylko 1 drzewo wyprowadzenia.
    
    \item Problem niepustości przecięcia języków bezkontekstowych \\
    pytanie, czy przecięcie dwóch języków bezkontekstowych $L_1$ i $L_2$ jest niepuste.
    
    \item Problem odpowiedniości Posta\\
    dla danego ciągu par słów $(w_1, v_1), (w_2, v_2), \ldots, (w_n, n_n)$, pytanie, czy istnieje taki ciąg indeksów $i_1, i_2, \ldots, i_m$ że $ w_{i_1} \ldots w_{i_m} = v_{i_1} \ldots v_{i_m}$.
    
    \begin{itemize}
        \item ograniczony - szczególny przypadek, gdy wymuszamy, że $i_1=1$.
    \end{itemize}

    \item Niepustość języka rozpoznawanego przez maszynę Turinga $M$ \\
    pytanie, $L(M) \neq \emptyset$.
    
    \item Stop deterministycznego automatu z dwoma licznikami - $A$ - deterministyczny automat z 2 licznikami $c_1, c_2$ bez wejścia. Pytanie, czy $A$ zatrzyma się w konfiguracji $(q_0, c_1 = 0. c_2=0)$.
\end{itemize}

\newpage

\subsection{Co-częściowo rozstrzygalne}
\begin{itemize}
    \item $\text{Co-Halt}_w$ - pytanie, czy dana maszyna Turinga $M$ nie zatrzyma się na słowie wejściowym $w$.
    \begin{itemize}
        \item szczególny przypadek, $\text{Co-Halt}_{\epsilon}$ - pytanie, czy maszyna Turinga $M$ nie zatrzyma się na pustym słowie wejściowym.
    \end{itemize}
    \item Problem uniwersalności/totalności języków bezkontekstowych - pytanie, czy język bezkontekstowy $L = \Sigma^*$.
\end{itemize}

\end{document}