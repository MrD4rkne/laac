\documentclass{article}
\setlength{\headheight}{8pt}

\usepackage[a4paper, margin=0.3in]{geometry}
\usepackage{lmodern}
\usepackage[polish]{babel}
\usepackage[T1]{fontenc}
\usepackage{amsmath, amssymb, amsthm}
\usepackage{hyperref}
\usepackage[shortlabels]{enumitem}
\usepackage{physics}
\usepackage{multicol}
\usepackage{titlesec}
\usepackage{tikz}
\usetikzlibrary{automata, positioning}

% Definicje środowisk do twierdzeń, lematów, dowodów itp.
\newtheorem{theorem}{Twierdzenie}[section]
\newtheorem{lemma}[theorem]{Lemat}
\newtheorem{corollary}[theorem]{Wniosek}
\newtheorem{proposition}[theorem]{Stwierdzenie}
\theoremstyle{definition}
\newtheorem{definition}{Definicja}[section]
\theoremstyle{remark}
\newtheorem*{remark}{Uwaga}
\newtheorem*{example}{Przykład}
\newtheorem*{property}{Własność}

\title{Praca domowa 1}
\author{Marcin Szopa}
\date{\today}

% Remove padding at the top of subsections
\titlespacing*{\subsection}{0pt}{*0}{*0}

% Redefine section numbering to use letters
\renewcommand{\thesection}{\Alph{section}}

% Define a command for singleton sets
\newcommand{\singleton}[1]{\left\{ #1 \right\}}

\begin{document}

\maketitle

\begin{remark}
    W całej pracy przyjmujemy, że pracujemy na językach nad skończonym alfabetem \(\Sigma\), tz. w \( \Sigma^* \). \(\mathcal{A}\) to klasa automatów rozpoznających te języki.
\end{remark}

\section{}

\subsection{Automat rozpoznający K \# L}

\begin{definition}

    Zdefiniujmy automat rozpoznający \(K \# L\). Niech \(M_1 = (\Sigma, Q_1,
    \singleton{q_1} F_1, \delta_1)\), \(M_2 = (\Sigma, Q_2, \singleton{q_2}, F_2,
    \delta_2)\) będą DFA rozpoznającymi odpowiednio języki \(K\) i \(L\). Istnieją,
    bo \(K\) i \(L\) regularne. Zakładamy, że \(Q_1 \cap Q_2 = \emptyset\) (zawsze
    można przemianować stany). Wtedy automat rozpoznający \(K \# L\) to \(M =
    (\Sigma, Q, \singleton{q}, F, \delta)\), gdzie:

    \begin{itemize}
        \item \(Q = Q_1 \times Q_2 \times \left\{ 0, 1 \right\}\),
        \item \(q = (q_1, q_2, 0)\),
        \item \(F = F_1 \times F_2 \times \left\{ 0, 1 \right\} \),
        \item \(\delta = \left\{ \left( \left(q_1, q_2, 0 \right), a, s \right): q_1 \in Q_1, q_2 \in Q_2, a \in \Sigma, s \in P(q_1, q_2) \right\}
              \cup \left\{ \left( \left(q_1, q_2, 1 \right), a, s\right): q_1 \in Q_1, q_2 \in Q_2, a \in \Sigma, s \in NP(q_1, q_2) \right\} \), gdzie

              \(P(q_1, q_2) = \left( \mathcal{F}(\delta_1, q_1) \times \singleton{\delta_2(q_2,a)}
              \cup \singleton{\delta_1(q_1,a)} \times \mathcal{F}(\delta_2, q_2) \right) \times \singleton{1}\),

              \(NP(q_1, q_2) =  \left( \delta_1(q_1,a), \delta_2(q_2,a), 0 \right)\),

              \(\mathcal{F}(\delta, q) = \left\{ \delta(q, a): a \in \Sigma \right\}\).
    \end{itemize}

    Inaczej mówiąc, automat \(M\) symuluje jednocześnie przejścia w \(M_1\) i
    \(M_2\). Przechodzimy między stanami na pozycji parzystej / nieparzystej (czyli
    parzystość litery po której dotarliśmy do danego stanu). Z pozycji nieparzystej
    przechodzimy w obu automatach po krawędzi ze wskazaną literą (są
    deterministyczne, więc jest dokładnie jedna taka krawędź). Z pozycji parzystej
    w jednym automacie przechodzimy po krawędzi z literą, a w drugim po dowolnej
    (co wprowadza niedeterminizm).
\end{definition}

\subsection{Dowód}

Pokażemy, że \(\mathcal{L}(M) = K \# L\).

\begin{proof}
    \(\rightarrow\)

    Zauważmy, że zgodnie z naszą konstrukcją:
    \begin{itemize}
        \item zaczynamy na pozycji \(\left(q_{1,1}, q_{2,1}, 0 \right)\),
        \item automat symuluje przejścia po \(M_1\) i \(M_2\), co jest równoważne przejściom
              po słowach \(b \in K\) i \(c \in L\),
        \item gdy przechodzimy ze stanu typu \(0\) do stanu typu \(1\), to w jednym automacie
              przechodzimy po krawędzi z aktualną literą, a w drugim po dowolnej, co jest
              równoważne \(a_{2i+1} \in \left\{ b_{2i+1}, c_{2i+1} \right\}\),
        \item gdy przechodzimy ze stanu typu \(1\) do stanu typu \(0\), to w obu automatach
              przechodzimy po krawędzi z aktualną literą, co jest równoważne \( a_{2i} =
              b_{2i} = c_{2i} \),
        \item kończymy na pozycji \(\left(q_{1,n+1} q_{2,n+1}, 0/1 \right)\), gdzie \(n =
              \abs{w}\).
    \end{itemize}

    Jako, że tak naprawdę przechodzimy po (dwuwymiarowych) automatach \(M_1\) i
    \(M_2\), to skoro ostatni stan jest akceptujący, to w obu automatach jesteśmy w
    stanie akceptującym. Dodatkowo, w każdym z automatów zrobiliśmy dokładnie
    \(\abs{w}\) kroków. Co oznacza, że istnieją słowa \(b \in K\) i \(c \in L\), że
    \(w\) da się podzielić na fragmenty, które tworzą \(w\) zgodnie z definicją \(K
    \# L\).

    Formalnie, niech \(w = a_1 a_2 a_3\ldots a_n \in \mathcal{L}(M)\).

    To znaczy, że istnieje bieg akceptujący \( \left(q_{1,1}, q_{2,1}, 0 \right)
    \xrightarrow{a_1} \left(q_{1,2}, q_{2,2}, 1 \right) \xrightarrow{a_2} \ldots
    \xrightarrow{a_n} \left(q_{1,n+1}, q_{2,n+1}, \left[ n \equiv 1 \mod 2 \right]
    \right) \). Dodatkowo na potrzeby wytłumaczenia będziemy uzupełniać ciąg \(d_1,
    d_2, \ldots, d_n \in \left\{ 1,2, 3 \right\}\), który oznacza, czy wzieliśmy
    literkę ze słowa z automatu \(M_1\) czy \(M_2\), czy z obu. Będziemy zachowywać
    niezmiennik, że dla danego \(j\) bieg jest poprawny w \(M_1\) / \(M_2\) o prefixie
    zgodnym z \(w\) po zastosowaniu \(d_n\), tz. dla danego \(j\): \(\text{dla
        \(k\)=1/2} \ \exists_{v \in M_k} \forall_{l \in \{ 1,2, \ldots j \}} d_l \in
    \{ k, 3 \} \implies b_l/c_l = w_l\).

    Załóżmy, że niezmiennik jest zachowany dla wszystkich \(j' \leq 2i\). Niech \(j=2i+1 \in
    \mathcal{N}_+, 2i+2 \leq n,\). Jesteśmy w stanie \( \left(q_{1,j}, q_{2,j}, 0
    \right) \). Skoro bieg jest akceptujący, to zgodnie z konstrukcją \(\delta\),
    \( \exists _{k \in \{1,2\}} \delta_k(q_{k,j}, a_j) = q_{k, j+1} \). Gdyby nie istniało, to 
    przejście do stanu \( \left(q_{1,j+1}, q_{2,j+1},1\right) \) by nie istniało.
    Czyli przypisujemy \(d_j = k\). Skoro niezmiennik do tej pory był zachowany dla tego \(M_k\),
    to nadal jest zachowany, bo \(d_j = k\).
    Dodatkowo, dla \(k \neq k' \in \{1,2\} \exists_{b \in \Sigma} (q_{k,j}, b, q_{k',j+1}) \in \delta\).
    Gdyby nie istniało, to zgodnie z konstrukcją \(M\), bieg w \(M\) by nie istniał. Co jest oczywiście sprzeczne.
    Zatem zachowaliśmy niezmiennik dla \(j\).

    Załóżmy, że niezmiennik jest zachowany dla \(j' \leq 2i+1\). Niech \(j=2i+2 \in
    \mathcal{N}, 2i+1 \leq n,\). Jesteśmy w stanie \( \left(q_{1,j}, q_{2,j}, 0
    \right) \). Skoro bieg jest akceptujący, to zgodnie z konstrukcją \(\delta\),
    \( \forall _{k \in \{1,2\}} \delta_k(q_{k,j}, a_j) = q_{k, j+1} \). Czyli
    przypisujemy \(d_j = 3\). Z czego wynika, że niezmiennik dla \(j\) jest zachowany.
    Gdyby nie był, to bieg nie byłby akceptujący (jak wyżej). 
    Czyli zachowaliśmy niezmiennik dla \(j\).
    
    Zauważmy, że dla \(j = 0\) niezmiennik jest zachowany, bo \(\{1, 2, \ldots j\} = \emptyset\).
    Dodatkowo, stany \(q_{1, n+1}, q_{2, n+1}\) są akceptujące odpowiednio w \(M_1\) i \(M_2\).

    Zatem na mocy indukcji wiemy, że istnieją słowa:
    \[ b_1 b_2 \ldots b_n \in K, c_1 c_2 \ldots c_n \in L \text{, że } \ \forall_{i \in \{1,2,\ldots n\}} \left( d_i = 3 \implies a_i = b_i = c_i \right) \land \left( d_i = 1/2 \implies b_i/c_i = w_i. \right) \]

    Zatem \(w \in K \# L\).

\end{proof}

\begin{proof}
    \(\leftarrow\)

    Niech \(w = a_1 a_2 a_3\ldots a_n \in K \# L\). Zgodnie z definicją istnieją
    \(b=b_1 b_2 \ldots b_n \in K\) i \(c=c_1 c_2 \ldots c_n \in L\) takie, że
    \(a_{2i} = b_{2i} = c_{2i}\) i \(a_{2i+1} \in \left\{ b_{2i+1}, c_{2i+1}
    \right\}\).

    Udowodnimy, że \(M\) akceptuje \(w\).

    Skoro \(b \in K \land c \in L\), to istnieją biegi akceptujące postaci
    \(q_{1,1} \xrightarrow{b_1} q_{1,2} \xrightarrow{b_2} \ldots \xrightarrow{b_n}
    q_{1,n+1}\), \(q_{2,1} \xrightarrow{c_1} q_{2,2} \xrightarrow{c_2} \ldots
    \xrightarrow{c_n} q_{2,n+1}\) odpowiednio w \(M_1\), \(M_2\).

    Zauważmy, że zgodnie z konstrukcją \(M\), stan \( \left( q_{1,n+1}, q_{2,n+1},
    \left[ n \equiv 1 \mod 2 \right] \right) \in F\).

    Skoro \(b_{2i} = c_{2i}\), to bieg postaci \( \left(q_{1,1}, q_{2,1}, 0 \right)
    \xrightarrow{b_1} \left(q_{1,2}, q_{2,2}, 1 \right) \xrightarrow{b_2} \ldots
    \xrightarrow{b_n} \left(q_{1,n+1}, q_{2,n+1}, \left[ n \equiv 1 \mod 2 \right]
    \right) \) i \( \left(q_{1,1}, q_{2,1}, 0 \right) \xrightarrow{c_1}
    \left(q_{1,2}, q_{2,2}, 1 \right) \xrightarrow{c_2} \ldots \xrightarrow{c_n}
    \left(q_{1,n+1}, q_{2,n+1}, \left[ n \equiv 1 \mod 2 \right] \right) \) są
    akceptujące w \(M\). Zatem \(b, c \in \mathcal{L}(M)\).

    Teraz pokaże, że można zmodyfikować któryś z tych biegów, aby uzyskać bieg
    akceptujący dla \(w\).

    Skoro oba biegi są akceptujące i składają się z tych samych stanów. to \(
    \forall_i \in \{1,2,\ldots n\} a_i = b_i \lor a_i = c_i \). Dalej, \( \forall_i
    \in \{1,2,\ldots n\} \left( \left( q_{1,i}, q_{2,i}, i \equiv 0 \mod 2 \right),
    a_i, \left( q_{1,i+1}, q_{2,i+1}, i \equiv 1 \mod 2 \right) \right) \in \delta
    \). Stąd, bieg \( \left(q_{1,1}, q_{2,1}, 0 \right) \xrightarrow{a_1}
    \left(q_{1,2}, q_{2,2}, 1 \right) \xrightarrow{a_2} \ldots \xrightarrow{a_n}
    \left(q_{1,n+1}, q_{2,n+1}, \left[ n \equiv 1 \mod 2 \right] \right) \) jest
    poprawny i akceptujący w \(M\).

    Zatem \(w \in \mathcal{L}(M)\).
\end{proof}

Skoro \(\mathcal{L}(M) = K \# L\), to \(K \# L\) jest regularny, bo \(M\) jest
automatem skończonym.

\section{Kontrprzykład}

Rozważmy \(K = \emptyset, L = \text{dowolny język nieregularny}\). Wtedy \(K \#
L = \emptyset\), który jest językiem regularnym. Jednak \(L\) jest
nieregularny. Zatem jest to kontrprzykład, co przeczy zdaniu z polecenia.

\section{Kontrprzykład}

Będziemy pracować na językach nad alfabetem \(\Sigma = \{a, b\}\).

Niech \(K_n = ( (a+b)a )^* (aa)^n, n \in \mathcal{N_+}\). Jest to język słów o
parzystej długości, które na parzystych pozycjach mają literę `a', a patrząc
tylko na pozycje nieparzyste, kończą się na \(a^n\). DFA rozpoznający \(K_n\)
ma \(\mathcal{O}(n)\) stanów.

Niech \(M_K\) to automat rozpoznający \(K_n\). \(M_K\) ma postać:

 \begin{center}
     \begin{tikzpicture}[shorten >=1pt, node distance=2cm, auto]
        \node[state, initial] (q_00)   {$q_{0,0}$};
        \node[state] (q_03) [above of=q_00] {$q_{0,3}$};
         \node[state] (q_01) [right of=q_00] {$q_{0,1}$};

         \node[state] (q_10) [right of=q_01] {$q_{1,0}$};
         \node[state] (q_11) [right of=q_10] {$q_{1,1}$};

         \node[state] (q_20) [right of=q_11] {$q_{2,0}$};
         \node[state] (q_21) [right of=q_20] {$q_{2,1}$};

         \node[state] (q_n01) [right of=q_21] {$q_{n-1,0}$};
         \node[state] (q_n11) [right of=q_n01] {$q_{n-1,1}$};
         \node[state, accepting] (q_n0) [right of=q_n11] {$q_{n,0}$};
         \node[state] (q_n3) [above of=q_n0] {$q_{n+1,1}$};

        \path[->]
             (q_00) edge [bend left] node {b} (q_03)
             (q_03) edge [bend left] node {a} (q_00)

            (q_00) edge node {a} (q_01)
            (q_01) edge node {a} (q_10)
            (q_10) edge node {a} (q_11)
            (q_10) edge [above, bend right] node {b} (q_03)

            (q_11) edge node {a} (q_20)
            (q_20) edge [above, bend right] node {b} (q_03)
            (q_20) edge node {a} (q_21)
            
            (q_21) edge [dotted] node {a} (q_n01)
            (q_n01) edge node {a} (q_n11)
            (q_n01) edge [above, bend right] node {b} (q_03)
            (q_n0) edge node {a} (q_n3)
            
            (q_n11) edge node {a} (q_n0)
            (q_n0) edge [above, bend right] node {b} (q_03)
            (q_n3) edge [bend left] node {a} (q_n0);
     \end{tikzpicture}
 \end{center}

\begin{remark}
    Na diagramie brakuje `śmietnika', czyli wszystkie brakujące krawędzie z literą \(b\), np. z \(q_{0,3},q_{0,1},q_{1,1}, q_{n+1,1}\).
    Śmietnik to stan, z którego nie można wyjść, a który nie jest akceptujący. Nie zmienia on rzędu rozmiaru automatu.
\end{remark}

Udowodnijmy, że faktycznie \(\mathcal{L}(M_K) = K_n\).

\begin{proof}
    \(\rightarrow\)

    Niech \(w \in K_n\). Wiemy z definicji, że \(w\) jest słowem o parzystej długości oraz 
    \(w = vu\), gdzie \(v \in \{(a+b)a\}^*\) i \(u = (aa)^n\).

    Gdy automat czyta \(v\), to nigdy nie przechodzimy do śmietnika, gdyż na pozycjach parzystych w \(w\) zawsze występuje `a', zatem też w \(v\).
    Zauważmy, że po wczytaniu \(v\) musimy być w stanie o drugiej współrzędnej parzystej oraz że wczytanie jednej litery prowadzi
    do stanu o innej parzystości drugiej współrzednej (z wyłączniem śmietnika).

    Skoro jesteśmy na pozycji parzystej, to wczytanie \(u\) prowadzi do \(q_{n,0}\).
    Zatem \(w\) prowadzi do stanu akceptującego. Stąd \(w \in \mathcal{L}(M_K)\).
\end{proof}

\begin{proof}
    \(\leftarrow\)

    Niech \(w \in \mathcal{L}(M_K)\). Wiemy z poprzednich obserwacji, że \(w\) jest słowem o parzystej długości oraz że 
    kończy się na \((aa)^n\). 
    Skoro \(w\) prowadzi do stanu akceptującego, to nie przechodzi przez śmietnik.
    Zauważmy, że \(w\) można podzielić na fragmenty \(v\) i \(u\), gdzie \(v \in \{(a+b)a\}^*\) i \(u = (aa)^n\).
    Stąd \(w\) pasuje do postaci regularnej \(K_n\). \(w \in K_n\).

\end{proof}


Niech \(L_n = ( (a+b)a )^* aa (ba)^{n-1}, n \in \mathcal{N_+}\). Jest to język
słów o parzystej długości, które na parzystych pozycjach mają literę `a', a
patrząc tylko na pozycje nieparzyste, kończą się na \( (b)^{n-1}\). DFA rozpoznający \(L_n\)
ma \(\mathcal{O}(n)\) stanów. Dowód jak dla \(K_n\).

Weźmy język \(K \# L\). Zauważmy, że \( \forall w= w_0 w_1 \ldots w_n \in K \#
L \ \forall_{i \in \{ 2, 4, \ldots n \} } w_i = a\). Wynika to z z definicji
\(K\), \(L\) i konstrukcji \(K \# L\).

Udowodnimy, że \(K \# L \subset \{ a, b \}^*\) jest równy językowi, którego
\(2n-1\) tą litera od końca jest `a' oraz wszystkie litery na parzystych
pozycjach są `a'.

\begin{proof}

    \(\rightarrow\)

    Łatwo zauważyć, że \(K \# L\) jest językiem o parzystej długości, bo \(K\) i \(L\) są językami o parzystej długości.
    Zarówno \(K\) i \(L\) na parzystych pozycjach mają literę `a', więc \(w\) również.

    Niech \(w = w_1 w_2 \ldots w_{2m} \in K \# L\). \(2n \leq 2m\), bo wszystkie
    słowa w \(K\) mają długość conajmniej \(2n\). Skoro \(w\) istnieje to zgodnie z
    definicją \(K \# L\) istnieją \(b=b_1 b_2 \ldots b_n \in K\) i \(c=c_1 c_2
    \ldots c_n \in L\) takie, że \(w_{2i} = b_{2i} = c_{2i}\) i \(w_{2i+1} \in
    \left\{ b_{2i+1}, c_{2i+1} \right\}\).

    Jako, że (patrząc na pozycje nieparzyste) \(b\) kończy się na \(a^n\), a \(c\)
    kończy się na \(b^{n-1}\), to \(n\)-ta nieparzysta pozycja od końca \(w\) jest
    `a'.

    Dodatkowo, \( \forall_{i \in \{2m - 2n + 1, 2m - 2n + 1 + 2, \ldots, 2m - 1 \}}
    w_i \in \{ a,b \}\). Każda możliwość spełniająca ten warunek jest poprawna.

    Finalnie, zauważmy, że zarówno \(b\) jak i \(c\) zaczyna się od dowolnego
    prefikxu \( ((a+b)a)^{2m-2n} \). Tu również, każda możliwość spełniająca ten
    warunek jest poprawna.

    Zatem \(w \in \text{język nad alfabetem } \{ a, b \}\) o słowach parzystej
    długości, conajmniej \(2n\), że \(2n-1\)-ta litera od końca jest `a' oraz
    wszystkie litery na parzystych pozycjach parzystych są `a'.

\end{proof}

\begin{proof}

    \(\leftarrow\)

    Niech \(w = w_1 w_2 \ldots w_{2m}\in \text{język nad alfabetem } \{ a, b \}\) o
    słowach parzystej długości, conajmniej \(2n\), że \(2n-1\)-ta litera od końca
    jest `a' oraz wszystkie litery na parzystych pozycjach parzystych są `a':

    \begin{itemize} 
        \item \(w_{2i} = a\) dla \(i \in \{1, 2, \ldots m\}\),
        \item \(w_{2m-2n+1} = a\).
    \end{itemize}

    Zdefiniujmy \(b = w_1 a w_3 \ldots w_{2m-2n-1} a a^{2n}\), \(c = w_1 a w_3
    \ldots w_{2m-2n-1} a a a (ba)^{n-1}\). Zauważmy, że \(w\) można podzielić na
    fragmenty \(b\) i \(c\) zgodnie z definicją \(K \# L\). Zatem \(w \in K \# L\).

\end{proof}

Skoro udowodniliśmy zgodność definicji \(K \# L\) z opisanym językiem, to policzmy jego klasy abstrakcji.

Zauważmy, że dowolne \(w = a(a+b)^n \in K \# L\).

Weźmy dwa słowa \(w_1 = a(a+b)^n\) i \(w_2 = a(a+b)^n\), gdzie \(n \neq m\).
Skoro się różnią, to istnieje \(i \in \{2, 3 \ldots n+1\}\) takie, że \(w_1[i] \neq w_2[i]\).
BSO niech \(w_1[i] = a\) i \(w_2[i] = b\).
Zauważmy, że \(w_1 \cdot (a+b)^{i-1} \in K \# L\), ale \(w_2 \cdot (a+b)^{i-1} \not \in K \# L\). Takich par jest \( \mathcal{O}(2^n) \).
Zatem na mocy Tw. Myhill-Nerode'a, minimalny DFA rozpoznający \(K \# L\) ma conajmniej \( \mathcal{O}(2^n) \) stanów.

Stąd nie istnieje \(p(n)\) takie, że \(K \# L\) jest rozpoznawany przez DFA o \(p(n)\) stanach.
Co dowodzi fałszywości zdania z polecenia.

\end{document}