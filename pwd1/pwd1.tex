\documentclass{article}
\setlength{\headheight}{8pt}

\usepackage[a4paper, margin=0.3in]{geometry}
\usepackage{lmodern}
\usepackage[polish]{babel}
\usepackage[T1]{fontenc}
\usepackage{amsmath, amssymb, amsthm}
\usepackage{hyperref}
\usepackage[shortlabels]{enumitem}
\usepackage{physics}
\usepackage{multicol}
\usepackage{titlesec}

% Definicje środowisk do twierdzeń, lematów, dowodów itp.
\newtheorem{theorem}{Twierdzenie}[section]
\newtheorem{lemma}[theorem]{Lemat}
\newtheorem{corollary}[theorem]{Wniosek}
\newtheorem{proposition}[theorem]{Stwierdzenie}
\theoremstyle{definition}
\newtheorem{definition}{Definicja}[section]
\theoremstyle{remark}
\newtheorem*{remark}{Uwaga}
\newtheorem*{example}{Przykład}
\newtheorem*{property}{Własność}

\title{Praca domowa 1}
\author{Marcin Szopa}
\date{\today}

% Remove padding at the top of subsections
\titlespacing*{\subsection}{0pt}{*0}{*0}

% Redefine section numbering to use letters
\renewcommand{\thesection}{\Alph{section}}

% Define a command for singleton sets
\newcommand{\singleton}[1]{\left\{ #1 \right\}}

\begin{document}

\maketitle

\begin{remark}
    W całej pracy przyjmujemy, że pracujemy na językach nad skończonym alfabetem \(\Sigma\), tz. w \( \Sigma^* \). \(\mathcal{A}\) to klasa automatów rozpoznających te języki.
\end{remark}

\section{}

\subsection{Automat rozpoznający K \# L}

\begin{definition}

Zdefiniujmy automat rozpoznający \(K \# L\).
Niech \(M_1 = (\Sigma, Q_1, \singleton{q_1} F_1, \delta_1)\), \(M_2 = (\Sigma, Q_2, \singleton{q_2}, F_2, \delta_2)\)
będą DFA rozpoznającymi odpowiednio języki \(K\) i \(L\). Zakładamy, że \(Q_1 \cap Q_2 = \emptyset\) Wtedy automat rozpoznający \(K \# L\) to
\(M = (\Sigma, Q, \singleton{q}, F, \delta)\), gdzie:

\begin{itemize}
    \item \(Q = Q_1 \times Q_2 \times \left\{ 0, 1 \right\}\),
    \item \(q = (q_1, q_2, 0)\),
    \item \(F = F_1 \times F_2 \times \left\{ 0, 1 \right\} \),
    \item \(\delta = \left\{ \left( \left(q_1, q_2, 0 \right), a, s \right): q_1, q_2 \in Q, a \in \Sigma, s \in P(q_1, q_2) \right\}
    \cup \left\{ \left( \left(q_1, q_2, 1 \right), a, s\right): q_1, q_2 \in Q, a \in \Sigma, s \in NP(q_1, q_2) \right\} \), gdzie \\
    \(P(q_1, q_2) = \left( \mathcal{F}(\delta_1, q_1) \times \singleton{\delta_2(q_2,a)}
     \cup \singleton{\delta_1(q_1,a)} \times \mathcal{F}(\delta_2, q_2) \right) \times \singleton{1}\), \\
     \(NP(q_1, q_2) =  \left( \delta_1(q_1,a), \delta_2(q_2,a), 0 \right)\) \\
\end{itemize}

\end{definition}

\section{}

Rozważmy \(K = \left\{ \right\}, L = \text{dowolny język nieregularny}\). Wtedy
\(K \# L = \left\{ \right\}\), który jest językiem regularnym. Jednak \(L\)
jest nieregularny. Zatem jest to kontrprzykład.

\end{document}k