\documentclass{article}
\setlength{\headheight}{8pt}

\usepackage[a4paper, margin=0.3in]{geometry}
\usepackage{lmodern}
\usepackage[polish]{babel}
\usepackage[T1]{fontenc}
\usepackage{amsmath, amssymb, amsthm}
\usepackage{hyperref}
\usepackage[shortlabels]{enumitem}
\usepackage{physics}
\usepackage{multicol}

% Definicje środowisk do twierdzeń, lematów, dowodów itp.
\newtheorem{theorem}{Twierdzenie}[section]
\newtheorem{lemma}[theorem]{Lemat}
\newtheorem{corollary}[theorem]{Wniosek}
\newtheorem{proposition}[theorem]{Stwierdzenie}
\theoremstyle{definition}
\newtheorem{definition}{Definicja}[section]
\theoremstyle{remark}
\newtheorem*{remark}{Uwaga}
\newtheorem*{example}{Przykład}
\newtheorem*{property}{Własność}

\title{Praca domowa 1}
\author{Marcin Szopa}
\date{\today}

% Remove padding at the top of subsections
\usepackage{titlesec}
\titlespacing*{\subsection}{0pt}{*0}{*0}

\begin{document}

\maketitle

W całej pracy przyjmujemy, że pracujemy na językach nad alfabetem (skończonym) \(\Sigma\), tz. w \( \Sigma^* \).

\(\mathcal{A}\) to klasa automatów rozpoznających te języki.

\section{a}

Zdefiniumy \(F: \mathcal{A} \to \mathcal{A}\) taki,
że dla automatu \(\left(\Sigma, Q, I, F, \sigma\right)\) zwraca on automat \(\left(\Sigma, Q, I, F, \sigma'\right)\), gdzie

\begin{align*}
\sigma' = \sigma \cup \bigcup_{l \in \Sigma} \left\{ \left< p, l, q \right> \mid \exists_{\epsilon \neq a \in \Sigma}{\left< p, a, q \right> \in \sigma} \right\}
\end{align*}

Inaczej, po prostu każdą niepusta krawędź będzie akceptować dowolną literę.

\begin{lemma}
    \(w \in L(F(A)) \iff \exists_{v \in L(A)} \abs{w}=\abs{v}\)
\end{lemma}

`\(\rightarrow\)` \\
Niech \(w \in L(F(A))\). Wtedy istnieje ścieżka w \(F(A)\) akceptująca \(w\). Zauważmy, że \(\forall_{\left< p, a, q \right>}\) z tego biegu \( \exists_{b \neq \epsilon} \left< p, b, q \right> \in \sigma\). Zatem moglibyśmy zamienić litery z aktualnej ścieżki na litery na odpowiadających krawędziach z \(A\) i otrzymalibyśmy bieg akceptujący słowo \(v\). Nie zmieniła się liczba przejść, a każda zamiana z litery \(\neq \epsilon\) była też na literę \(\neq \epsilon\). Stąd \(\abs{v}=\abs{w}.\) \(\blacksquare\)

`\(\leftarrow\)` \\
Niech \(v \in L(A)\). Skoro \(F\) jedynie dodało przejścia między stanami, to \(v \in L(F(A))\), bo \(\sigma \subseteq \sigma'\). \(\blacksquare\)


\end{document}